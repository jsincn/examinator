\problem{Creating problems}

This section describes how the problems of an exam are created.
It is advisable to not only read this document, but also have a look at the source code of the sample problems.

\vspace{\baselineskip}
\textbf{Hint:} Problems that can be solved without solving previous subproblems first are marked by *. (It is up to you whether or not you want to tell that your students, but if you do, please do it carefully.)

\subproblem*{1}
First, we need to declare a problem.
\begin{solutionbox}[2cm]
	\verb|\problem{The problem title}|

	\correction{Here afterwards, we can insert arbitrary material like text or pictures.}
\end{solutionbox}

\subproblem{1}
Then, we need to declare the subproblems.
\begin{solutionbox}[3cm]
	\verb|\subproblem{1}| \quad or\quad\verb|\subproblem*{1}|

	\correction{The argument (number) is the amount of credits for the subproblem.}

	\correction{The star-version just creates a star next to the subproblem number.
	This could e.\,g. indicate subproblems wich are solvable independently.}
\end{solutionbox}

\vspace{2em}
There are different environments to create solutionboxes.

\subproblem{1}
How can we define a normal solutionbox of 3\,cm in height?
%FIXME: include the siuntix package by default with sensible options

\begin{solutionbox}[3cm]
	\begin{verbatim}
		\begin{solutionbox}[3cm]
		This is the solution.
		\end{solutionbox}
	\end{verbatim}
	\correction{The optional argument specifies the height of the solutionbox.}
\end{solutionbox}

\subproblem{1}
How can we annotate the solution with correction notes?

\begin{solutionbox}[3cm]
	\verb|\credit| \credit\\[0.5em]
	\verb|\hcredit| \hcredit\\[0.5em]
	\verb|\correction{| \correction{This is a correction note.} \verb|}|
\end{solutionbox}

\subproblem{1}
How can we define a solutionbox with instructions?

\begin{instructionbox}[5cm]
	\begin{verbatim}
		\begin{instructionbox}
			This is instructiontext, e.\g. a table preprint
			\solution{
	\end{verbatim}
	\solution{\texttt{This is the solution.}}
	\begin{verbatim}
		}
		\end{instructionbox}
	\end{verbatim}

	\correction{Note: instructionboxes allow no text input in the PDF as it would interfere with instructions}
\end{instructionbox}

\newpage

\subproblem{1}
How can we define a solutionbox with gridlines?

\begin{gridsolutionbox}[3cm]
	\begin{verbatim}
		\begin{gridsolutionbox}
			This is the solution.
		\end{gridsolutionbox}
	\end{verbatim}
\end{gridsolutionbox}

\subproblem{1}
How can we define a solutionbox with grid and instructions?

\begin{gridinstructionbox}[5cm]
	\begin{verbatim}
		\begin{gridinstructionbox}
			This is instructiontext, e.\g. a table preprint
			\solution{
	\end{verbatim}
	\solution{\texttt{This is the solution.}}
	\begin{verbatim}
		}
		\end{gridinstructionbox}
	\end{verbatim}

	\correction{Note: instructionboxes allow no text input in the PDF as it would interfere with instructions}
\end{gridinstructionbox}


\subproblem*{1}
\glqq{}Ich bitte Sie, das neue Corporate Design konsequent umzusetzen.
Sehen Sie darin eine Loyalitätspflicht der Hochschulmitglieder, damit sich der Hinweis auf die Rechtsverbindlichkeit erübrigen kann.\grqq{}\,\footnote{Wolfgang A.\ Herrmann, Corporate Design Handbuch 2016}

\vspace{\baselineskip}
For that reason, TUMexam is based on Helvetica, and we do our best to convince you of Helvetica--even in math mode:

\begin{align*}
	\text{mathnormal}
		&& abcdefghhjkklmnopqrstuvwxyz\\
		&& ABCDEFGHJKLMNOPQRSTUVWXYZ\\
		&& 1234567890\\
		&&
		\alpha\beta\gamma\delta\epsilon\zeta\eta\theta\kappa\lambda\mu\nu\xi\pi\rho\sigma\tau\phi\chi\psi\omega\\
		&& \Delta\Theta\Lambda\Pi\Phi\Psi\Omega\\
	\text{mathbit}
		&& \mathbit{abcdefghjklmnopqrstuvwxyz}\\
		&& \mathbit{ABCDEFGHIJKLMNOPQRSTUVWXYZ}\\
		&& \mathbit{01234567890}\\
		&& \mathbit{\alpha\beta\gamma\delta\epsilon\zeta\eta\theta\kappa\lambda\mu\nu\xi\pi\rho\sigma\tau\phi\chi\psi\omega}\\
		&& \mathbit{\Delta\Theta\Lambda\Pi\Phi\Psi\Omega}\\
	\text{mathcal}
		&& \mathcal{ABCDEFGHIJKLMNOPQRSTUVXYZ}\\
%	\text{mathcal+mathbit}
%		&& \mathcal{\mathbit{ABCDEFGHIJKLMNOPQRSTUVXYZ}}\\
%FIXME broken
	\text{symbols}
		&& \sum_{n=1}^N \frac{N(N-1)}{2}, \quad
		\frac{1}{2\pi}\int_{-\infty}^\infty f(x) e^{-j\omega t} \pd{t}\\
	\text{addons}
		&& \pr{X=x\,|\,Y=y}, \var{x}, \expect{X}
\end{align*}

If you plan to disobey this request, you may use computer modern instead by setting \verb|\seriftrue| in \verb|examconf.tex|.
