\problem{Multiple choice problems with TUMexam}

TUMexam supports the automatic evaluation of multiple choice problems (both single and multiple answer) through so called \emph{calculation functions} that are supplied as optional argument to the \verb|\subproblem| macro.


\subproblem[binary_mc]{1}
How can we define a multiple choice question?
\begin{multiplechoice}
	\item* \verb|\subproblem[<calculation_function>]{<max_credit>}|
	\item \verb|\subproblem{<max_credit>}| \correction{This does not work as it would be a normal subproblem.}
\end{multiplechoice}


\subproblem{5}
What are the available calculation functions?
\begin{solutionbox}[13cm]
\begin{itemize}
	\item \verb|mc|\\
	Every correct cross gives +1 credit, every incorrect cross gives -1 credit.
	Missing crosses do not count.
	The lowest possible amount of credits is 0.
	If the specified maximum credits do not equal the number of correct answers, credits are scaled by the factor \verb|max_credit/correct_options|.
	You \textbf{must} ensure that this fraction is a multiple of 0.5 credits.
	\item \verb|strict_mc|\\
	Works the same way as \verb|mc|, but a missing cross also gives -1 credit.
	\item \verb|binary_mc|\\
	Gives the specified maximum amount of credits if everything is correct and 0 credits otherwise.
	\item \verb|ternary_mc|\\
	The subproblem gives the specified amount of credits if everything is answered correctly.
	If there is exactly one cross missing \emph{or} one wrong answer crossed, the resulting amount of credits is \verb|max_credit/2|.
	Otherwise, the amount of credits is 0.

	Note that in a problem with two correct options, crossing one correct and one incorrect answer gives 0 credits since aside from the incorrect answer there is also one correct answer missing.
	\item \verb|per_option_mc|\\
	In this case, each answer option is graded individually depending on whether its state (crossed or not crossed) is correct.
	You \textbf{must} set the maxmimum number of credits to be a multiple of the total number of options, e.\,g.\ having 5 options leaves you with 5, 10, 15, \ldots as possible values.
	Each option with correct state is than awarded 1, 2, 3, \ldots credits.

	Example: A question with 2 correct options out of 5 options in total and a \verb|max_credit| of 5 would give 3 credits (!) if a student simply crosses nothing since in that case 3 options are ``correctly not crossed''.
	If \verb|max_credit| is set to 10, the result would be 6 credits in this example.
\end{itemize}
\end{solutionbox}


\subproblem[binary_mc]{1}
What is the command to typeset instructions on how to mark correct answers?
\begin{center}
	\mcnotes{}
\end{center}
\begin{multiplechoice}
	\item* \verb|\mcnotes{}|
	\item \verb|\notesmc{}|
	\item \verb|\makenotes{}|
	\item \verb|\makemc{}|
\end{multiplechoice}


